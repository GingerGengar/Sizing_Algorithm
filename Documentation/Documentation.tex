\documentclass[a4paper, 12pt]{report}

\usepackage{amsmath}
\usepackage{esint}
\usepackage{comment}
\usepackage{amssymb}
\usepackage{commath}
\usepackage{geometry}
\usepackage{graphicx}
\usepackage{hyperref}
\usepackage{listings}
\usepackage{xcolor}
\usepackage{array}
\usepackage{float}

\geometry{portrait, margin= 0.5in}
\setcounter{MaxMatrixCols}{30}

\definecolor{codegreen}{rgb}{0,0.6,0}
\definecolor{codegray}{rgb}{0.5,0.5,0.5}
\definecolor{codepurple}{rgb}{0.58,0,0.82}
\definecolor{backcolour}{rgb}{0.95,0.95,0.92}

\lstset{
	columns=fullflexible,
	frame=single,
	breaklines=true,
	backgroundcolor=\color{backcolour},   
	commentstyle=\color{codegreen},
	keywordstyle=\color{magenta},
	numberstyle=\tiny\color{codegray},
	stringstyle=\color{codepurple},
	basicstyle=\ttfamily\footnotesize,
	keepspaces=true,                 
	numbersep=5pt,                  
	showspaces=false,               
	showtabs=false,                  
	tabsize=2
}

\title{Sizing Algorithm Documentation}
\author{Team 7}
\date{$21^{st}$ June 2022}

\begin{document}
\maketitle
\tableofcontents
\newpage
\begin{center}


\section{Rationale}
We need to determine $2$ very important things:
\begin{enumerate}
\item The thrust to weight ratio affects:
\begin{itemize}
\item acceleration:
\item climb rate:
\item maximum speed:
\item turn rate:
\item Aircraft Endurance:
\end{itemize}
\item The stall speed is directly determined by the wing loading and the maximum lift coefficient. Wing Loading will affect:
\begin{itemize}
\item stall speed
\item induced drag
\item take-off distance
\end{itemize}
\end{enumerate}

\section{Sizing Methodology}
\begin{enumerate}
\item First Method:
    \begin{enumerate}
    \item Firstly use the stall speed to determine wing loading
        Stall speed should be independent of engine size
    \item After figuring out wing loading, determine the thrust to weight ratio based on 
        takeoff distance
        Rate of climb
    \end{enumerate}
\item Second Alternative Method:
    \begin{enumerate}
    \item Determine the Thrust to Weight Ratio
    \item Determine the wing loading by direct computation 
    \end{enumerate}
\end{enumerate}

$q$ is the dynamic pressure of an aircraft flying. It is defined as
$$q = \frac{1}{2}\rho v^{2}$$


The equation below was found in page $118$,
$$\frac{T}{W} = \left(\frac{\eta_{p}}{V}\right)\left(\frac{P}{W}\right)$$
wherein $\eta_{p}$ represents the propeller efficiency, 
$P$ represents power of the engines, $W$ represents the weight of the aircraft,
$V$ represents the true air speed of the aircraft.

Wing loading is basically the weight of the aircraft divided by the total wing area.
$$L_{wi} = \frac{W}{S} = \frac{1}{2}\rho v_{stall}^{2} c_{L,max}$$
Our typical would be around $c_{L,max} \approx 1.2->1.5$, 
Sweep only reduces your maximum coefficient of lift. 





\section{Textual References}
\begin{enumerate}
\item Page $117$ has table for thrust to weight of typical aircrafts.
\item Page $119$ has a table for power to weight ratio of typical aircrafts.
\item Page $124$ has a tabel for typical wing loadings.
\item Page $126$ has description on typical $c_{L,max}$ values.
\end{enumerate}


\begin{comment}
Take-off distance is shown in this equation for a propeller (Page $130$)
$$\frac{W}{S} = T_{op} \sigma c_{L,TO}(hp/W)$$
wherein $T_{op}$ represents
$\sigma$ represents the density ratio. This is basically just the air density $\rho$ at takeoff altitude divided by sea-level density
$c_{L,TO}$ represents the take-off lift coefficient. This is the maximum lift coefficient divided by $1.21 = 1.1^{2}$. This is because the aircraft takes off at around $1.1$ of its stall speed.
$hp$ represents 
$W$ represents take-off weight of the aircraft
\end{comment}


\end{center}
\end{document}
