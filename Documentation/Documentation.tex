\documentclass[letter, 12pt]{article}
%Seperator
%Seperator
%Seperator
%Seperator
%Seperator
%Seperator
%Seperator

%Packages used and configurations of the document
\usepackage{amsmath}
\usepackage{esint}
\usepackage{comment}
\usepackage{amssymb}
\usepackage{commath}
\usepackage{geometry}
\usepackage{graphicx}
\usepackage{hyperref}
\usepackage{listings}
\usepackage{xcolor}
\usepackage{array}
\usepackage{float}
\geometry{portrait, margin= 0.5in}
\setcounter{MaxMatrixCols}{30}

\definecolor{codegreen}{rgb}{0,0.6,0}
\definecolor{codegray}{rgb}{0.5,0.5,0.5}
\definecolor{codepurple}{rgb}{0.58,0,0.82}
\definecolor{backcolour}{rgb}{0.95,0.95,0.92}

\lstset{
	columns=fullflexible,
	frame=single,
	breaklines=true,
	backgroundcolor=\color{backcolour},   
	commentstyle=\color{codegreen},
	keywordstyle=\color{magenta},
	numberstyle=\tiny\color{codegray},
	stringstyle=\color{codepurple},
	basicstyle=\ttfamily\footnotesize,
	keepspaces=true,                 
	numbersep=5pt,                  
	showspaces=false,               
	showtabs=false,                  
	tabsize=2
}

%The Title of the document and its authors
\title{Aircraft Sizing Algorithm Documentation}
\author{Team 7:
\\Cooper LeComp, John A. Papas Dennerline, 
\\Ian Greene, Christos Levy, Daniel Qi, Hans Suganda}
\date{$27^{\text{th}}$ August 2022}


%Seperator
%Seperator
%Seperator
%Seperator
%Seperator
%Seperator
%Seperator

\begin{document}
\maketitle
\newpage
\tableofcontents
\newpage
\begin{center}

%Seperator
%Seperator
%Seperator
%Seperator
%Seperator
\section{Rationale}
\begin{comment}
\end{comment}
We need to determine $2$ very important things:
\begin{enumerate}
\item The thrust to weight ratio affects:
\begin{itemize}
\item acceleration:
\item climb rate:
\item maximum speed:
\item turn rate:
\item Aircraft Endurance:
\end{itemize}
\item The stall speed is directly determined by the wing loading and the maximum lift coefficient. Wing Loading will affect:
\begin{itemize}
\item stall speed
\item induced drag
\item take-off distance
\end{itemize}
\end{enumerate}

%Seperator
%Seperator
%Seperator
%Seperator
%Seperator
\section{Sizing Methodology}
\begin{comment}
\end{comment}
\begin{enumerate}
\item First Method:
    \begin{enumerate}
    \item Firstly use the stall speed to determine wing loading
        Stall speed should be independent of engine size
    \item After figuring out wing loading, determine the thrust to weight ratio based on 
        takeoff distance
        Rate of climb
    \end{enumerate}
\item Second Alternative Method:
    \begin{enumerate}
    \item Determine the Thrust to Weight Ratio
    \item Determine the wing loading by direct computation 
    \end{enumerate}
\end{enumerate}

%Seperator
%Seperator
%Seperator
%Seperator
%Seperator
\section{Common Definitions}
\begin{comment}
\end{comment}
$q$ is the dynamic pressure of an aircraft flying. It is defined as
$$q = \frac{1}{2}\rho v^{2}$$
wherein $\rho$ represents the density of the air, $v$ represents the velocity the aircraft is flying at.
The \texttt{Matlab} script below defines the global variables that is needed for this entire simulation,
\lstinputlisting[language=Matlab]{../Global_Variables.m}
$$$$
The script below handles the global variable initializations,
\lstinputlisting[language=Matlab]{../Initializations.m}

%Seperator
%Seperator
%Seperator
%Seperator
%Seperator
\section{Stall Speed Constraint}
\begin{comment}
\end{comment}
Wing loading is basically the weight of the aircraft divided by the total wing area.
$$L_{wi} = \frac{W}{S} = \frac{1}{2}\rho v_{stall}^{2} c_{L,max}$$
%$$L_{wi} = \frac{1}/{2}*\rho*(v_{stall}^{2})* c_{Lmax}$$
%$$L_wi = 1/2*rho*(v_stall^2)*c_Lmax$$
Our typical would be around $c_{L,max} \approx 1.2 \to 1.5$, 
Sweep only reduces your maximum coefficient of lift. The \texttt{Matlab} implementation is shown below,
\lstinputlisting[language=Matlab]{../Wing_Loading.m}

%Seperator
%Seperator
%Seperator
%Seperator
%Seperator
\section{Climb Rate Constraint}
\begin{comment}
\end{comment}
The cimb rate $G$ is defined as the ratio between vertical and horizontal distance travelled when the aircraft is climbing
$D$ is going to represent drag. 
\begin{equation}\frac{D}{W} = \frac{T}{W } - G \label{Kinematic Relation}\end{equation}
There is also another relation for drag to weight ratio,
\begin{equation}\frac{D}{W } = \frac{qc_{D,0}}{L_{wi}} + L_{wi}\frac{1}{q\pi A_{r}e} \label{Drag to Weight Ratio}\end{equation}
wherein $A_{r}$ is the aspect ratio of the wings, and $e$ represents the efficiency factor due to deviating from the elliptic lift distribution.
Substituting equation $\ref{Kinematic Relation}$ into equation $\ref{Drag to Weight Ratio}$ and solving for $\displaystyle\frac{T}{W }$,
$$\frac{D}{W } = \frac{T}{W } - G  = \frac{qc_{D,0}}{L_{wi}} + L_{wi}\frac{1}{q\pi A_{r}e} $$
$$\frac{T}{W }  = \frac{qc_{D,0}}{L_{wi}} + L_{wi}\frac{1}{q\pi A_{r}e} + G $$
wherein $c_{D,0}$ represents the zero-lift drag coefficient of an aircraft. For a clean propeller aircraft, $c_{D,0} = 0.02$. For a dirty propeller aircraft, then $c_{D,0} = 0.03$.
%$$\frac{T}{W }  = \frac{(q*c_{D,0})}/{(L_{wi})} + \frac{L_{wi}}/{(q*\pi*A_{r}*e)} + G$$
%$$T_W   = (q*c_D,0)/(L_wi) + L_wi/(q*pi*A_r*e) + G$$
The expression for thrust to weight ratio computations are implemented below,
\lstinputlisting[language=Matlab]{../Thrust_Weight_Ratio.m}



%Seperator
%Seperator
%Seperator
%Seperator
%Seperator
\section{Estimation of Motor Power Rating}
\begin{comment}
\end{comment}
The equation below was found in page $118$,
$$\frac{T}{W} = \left(\frac{\eta_{p}}{V}\right)\left(\frac{P}{W}\right)$$
wherein $\eta_{p}$ represents the propeller efficiency, 
$P$ represents power of the engines, $W$ represents the weight of the aircraft,
$V$ represents the true air speed of the aircraft.

%Seperator
%Seperator
%Seperator
%Seperator
%Seperator
\section{Textual References}
\begin{comment}
\end{comment}
\begin{enumerate}
\item Page $117$ has table for thrust to weight of typical aircrafts.
\item Page $119$ has a table for power to weight ratio of typical aircrafts.
\item Page $124$ has a tabel for typical wing loadings.
\item Page $126$ has description on typical $c_{L,max}$ values.
\item Page $135$ has descriptions of what $c_{D,0}$ should be.
\end{enumerate}

\begin{comment}
Take-off distance is shown in this equation for a propeller (Page $130$)
$$\frac{W}{S} = T_{op} \sigma c_{L,TO}(hp/W)$$
wherein $T_{op}$ represents
$\sigma$ represents the density ratio. This is basically just the air density $\rho$ at takeoff altitude divided by sea-level density
$c_{L,TO}$ represents the take-off lift coefficient. This is the maximum lift coefficient divided by $1.21 = 1.1^{2}$. This is because the aircraft takes off at around $1.1$ of its stall speed.
$hp$ represents 
$W$ represents take-off weight of the aircraft
\end{comment}


\end{center}
\end{document}
